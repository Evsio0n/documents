\chapter{Device Tree}

\section{在kernel启动时}

在 kernel 的 main.c 中的 setup\_arch ,在设置cpu 后, 
进行设备树的 设置, 如果 setup\_machine\_fdt 返回 NULL 
则会去尝试解析  tags.
\begin{lstlisting}
mdesc = setup_machine_fdt(__atags_pointer);
if (!mdesc)
	mdesc = setup_machine_tags(__atags_pointer, __machine_arch_type);
    machine_desc = mdesc;
	machine_name = mdesc->name;
\end{lstlisting}




接下来的 一些 文字我希望分析 启动过程中 和 device tree 相关的事情。现在看来 是要 和 setup\_machine\_fdt
耗上了。


\subsection{setup\_machine\_fdt}

\begin{lstlisting}
    /**
    * setup_machine_fdt - Machine setup when an dtb was passed to the kernel
    * @dt_phys: physical address of dt blob
    *
    * If a dtb was passed to the kernel in r2, then use it to choose the
    * correct machine_desc and to setup the system.
    */
   const struct machine_desc * __init setup_machine_fdt(unsigned int dt_phys)
   {
       const struct machine_desc *mdesc, *mdesc_best = NULL;
   
   #ifdef CONFIG_ARCH_MULTIPLATFORM
       DT_MACHINE_START(GENERIC_DT, "Generic DT based system")
       MACHINE_END
   
       mdesc_best = &__mach_desc_GENERIC_DT;
   #endif
   
       if (!dt_phys || !early_init_dt_verify(phys_to_virt(dt_phys)))
           return NULL;
   
       mdesc = of_flat_dt_match_machine(mdesc_best, arch_get_next_mach);
   
       if (!mdesc) {
           const char *prop;
           int size;
           unsigned long dt_root;
   
           early_print("\nError: unrecognized/unsupported "
                   "device tree compatible list:\n[ ");
   
           dt_root = of_get_flat_dt_root();
           prop = of_get_flat_dt_prop(dt_root, "compatible", &size);
           while (size > 0) {
               early_print("'%s' ", prop);
               size -= strlen(prop) + 1;
               prop += strlen(prop) + 1;
           }
           early_print("]\n\n");
   
           dump_machine_table(); /* does not return */
       }
   
       /* We really don't want to do this, but sometimes firmware provides buggy data */
       if (mdesc->dt_fixup)
           mdesc->dt_fixup();
   
       early_init_dt_scan_nodes();
   
       /* Change machine number to match the mdesc we're using */
       __machine_arch_type = mdesc->nr;
   
       return mdesc;
   }
   
\end{lstlisting}


从注释上看 这个函数是要在  dtb 传入 内核的时候,来根据dtb 做一些相应设置的。  


在一开始 ,  如果发现 dt\_phys 为空, 即没有 dtb 传递进来, 那么,直接返回NULL , 返回后会又 tags 的 方法去 用 tags 的 方法进行设置。 
\begin{lstlisting} 
early_init_dt_verify(phys_to_virt(dt_phys)

bool __init early_init_dt_verify(void *params)
{
	if (!params)
		return false;

	/* check device tree validity */
	if (fdt_check_header(params))
		return false;

	/* Setup flat device-tree pointer */
	initial_boot_params = params;
	of_fdt_crc32 = crc32_be(~0, initial_boot_params,
				fdt_totalsize(initial_boot_params));
	return true;
}

\end{lstlisting}
上面的 code 先 验证 device tree 是不是有效。 然后用  dtb 的 虚拟地址 赋值为initial\_boot\_params \label{a1} .



\subsubsection{of\_flat\_dt\_match\_machine}

接下来:

\begin{lstlisting}

static const void * __init arch_get_next_mach(const char *const **match)
{
	static const struct machine_desc *mdesc = __arch_info_begin;
	const struct machine_desc *m = mdesc;

	if (m >= __arch_info_end)
		return NULL;

	mdesc++;
	*match = m->dt_compat;
	return m;
}


mdesc = of_flat_dt_match_machine(mdesc_best, arch_get_next_mach);
\end{lstlisting}

of\_flat\_dt\_match\_machine从 表中找到 匹配的 machine . 传入的参数是 :
\begin{itemize}
    \item default\_match:  这个位置传入的是 mdesc\_best  如果没有在匹配表中找到 匹配的,就会用这个
    返回。
    \item get\_next\_compat : 传入的是arch\_get\_next\_mach  回调函数, 这个回调函数
    用来找到下一个 合适的 匹配表。
    
\end{itemize}


arch\_get\_next\_mach 中 \_\_arch\_info\_begin 到 \_\_arch\_info\_end 依次 返回machine\_desc 的 dt\_compat
成员。那么 这个区段 究竟 存了些什么, 谁 存的 呢? 接下来 就要把 这件事情搞得很明白。


在链接脚本中,存在这样的区段:
\begin{lstlisting}
    .init.arch.info : {
        __arch_info_begin = .;
        *(.arch.info.init)
        __arch_info_end = .;
    }  
\end{lstlisting}

从这里可以得知, 为了进行接下来的研究, 需要找到  把 attribute 设置为 在 section ".arch.info.init" 的代码。

而 这段代码为 :
\begin{lstlisting}
    arch/arm/include/asm/mach/arch.h

    /*
 * Set of macros to define architecture features.  This is built into
 * a table by the linker.
 */
#define MACHINE_START(_type,_name)			\
static const struct machine_desc __mach_desc_##_type	\
 __used							\
 __attribute__((__section__(".arch.info.init"))) = {	\
	.nr		= MACH_TYPE_##_type,		\
	.name		= _name,

#define MACHINE_END				\
};

#define DT_MACHINE_START(_name, _namestr)		\
static const struct machine_desc __mach_desc_##_name	\
 __used							\
 __attribute__((__section__(".arch.info.init"))) = {	\
	.nr		= ~0,				\
	.name		= _namestr,

#endif

\end{lstlisting}



在 i.mx6ull 中 有:
\begin{lstlisting}
    arch/arm/mach-imx/mach-imx6ul.c :

    static const char *imx6ul_dt_compat[] __initconst = {
        "fsl,imx6ul",
        "fsl,imx6ull",
        NULL,
    };
    
    DT_MACHINE_START(IMX6UL, "Freescale i.MX6 Ultralite (Device Tree)")
        .map_io		= imx6ul_map_io,
        .init_irq	= imx6ul_init_irq,
        .init_machine	= imx6ul_init_machine,
        .init_late	= imx6ul_init_late,
        .dt_compat	= imx6ul_dt_compat,
    MACHINE_END
      
\end{lstlisting}


看到这些后, 我们返回来 分析 of\_flat\_dt\_match\_machine 中的 逻辑。dt\_root 代表了
设备树的根节点。   接下来 , 依次 从 .arch.info.init 段中,取出 machine\_desc 结构,然后 
提取出 其中的 dt\_compat 成员 和 传入的dtb 中的 比对。 

具体的匹配过程:

\begin{lstlisting}

    /**
    * of_fdt_match - Return true if node matches a list of compatible values
    */
   int of_fdt_match(const void *blob, unsigned long node,
                    const char *const *compat)
   {
       unsigned int tmp, score = 0;
   
       if (!compat)
           return 0;
   
       while (*compat) {
           tmp = of_fdt_is_compatible(blob, node, *compat);
           if (tmp && (score == 0 || (tmp < score)))
               score = tmp;
           compat++;
       }
   
       return score;
   }

/**
 * of_flat_dt_match - Return true if node matches a list of compatible values
 */
int __init of_flat_dt_match(unsigned long node, const char *const *compat)
{
	return of_fdt_match(initial_boot_params, node, compat);
}

int of_fdt_is_compatible(const void *blob,
		      unsigned long node, const char *compat)
{
	const char *cp;
	int cplen;
	unsigned long l, score = 0;

	cp = fdt_getprop(blob, node, "compatible", &cplen);
	if (cp == NULL)
		return 0;
	while (cplen > 0) {
		score++;
		if (of_compat_cmp(cp, compat, strlen(compat)) == 0)
			return score;
		l = strlen(cp) + 1;
		cp += l;
		cplen -= l;
	}

	return 0;
}

\end{lstlisting}

还记得 initial\_boot\_params 吗? 不记得请 initial\_boot\_params \ref{a1}  , of\_flat\_dt\_match 根据 compat 从跟节点 
判断 dtb 中的 compatible 字段 是否最合适。 这里判断是不是合适 使用了 score 的方法, 当完全匹配时,
score 为 1  并返回; 否则就是字符串部分匹配,返回非零值。 


所以 到这里,我们就可以明白, of\_flat\_dt\_match\_machine  返回了 一个 在  \_\_arch\_info\_begin 到 \_\_arch\_info\_end  中的 最合适的 struct machine\_desc
结构。 


\subsubsection{early\_init\_dt\_scan\_nodes}

\begin{lstlisting}
    void __init early_init_dt_scan_nodes(void)
    {
        /* Retrieve various information from the /chosen node */
        of_scan_flat_dt(early_init_dt_scan_chosen, boot_command_line);
    
        /* Initialize {size,address}-cells info */
        of_scan_flat_dt(early_init_dt_scan_root, NULL);
    
        /* Setup memory, calling early_init_dt_add_memory_arch */
        of_scan_flat_dt(early_init_dt_scan_memory, NULL);
     }    
\end{lstlisting}



\begin{lstlisting}
    

/**
 * of_scan_flat_dt - scan flattened tree blob and call callback on each.
 * @it: callback function
 * @data: context data pointer
 *
 * This function is used to scan the flattened device-tree, it is
 * used to extract the memory information at boot before we can
 * unflatten the tree
 */
int __init of_scan_flat_dt(int (*it)(unsigned long node,
				     const char *uname, int depth,
				     void *data),
			   void *data)
{
	const void *blob = initial_boot_params;
	const char *pathp;
	int offset, rc = 0, depth = -1;

        for (offset = fdt_next_node(blob, -1, &depth);
             offset >= 0 && depth >= 0 && !rc;
             offset = fdt_next_node(blob, offset, &depth)) {

		pathp = fdt_get_name(blob, offset, NULL);
		if (*pathp == '/')
			pathp = kbasename(pathp);
		rc = it(offset, pathp, depth, data);
	}
	return rc;
}
\end{lstlisting}


\begin{lstlisting}
int __init early_init_dt_scan_chosen(unsigned long node, const char *uname,
    int depth, void *data)
{
int l;
const char *p;

pr_debug("search \"chosen\", depth: %d, uname: %s\n", depth, uname);

if (depth != 1 || !data ||
(strcmp(uname, "chosen") != 0 && strcmp(uname, "chosen@0") != 0))
return 0;

early_init_dt_check_for_initrd(node);

/* Retrieve command line */
p = of_get_flat_dt_prop(node, "bootargs", &l);
if (p != NULL && l > 0)
strlcpy(data, p, min((int)l, COMMAND_LINE_SIZE));

/*
* CONFIG_CMDLINE is meant to be a default in case nothing else
* managed to set the command line, unless CONFIG_CMDLINE_FORCE
* is set in which case we override whatever was found earlier.
*/
#ifdef CONFIG_CMDLINE
#ifndef CONFIG_CMDLINE_FORCE
if (!((char *)data)[0])
#endif
strlcpy(data, CONFIG_CMDLINE, COMMAND_LINE_SIZE);
#endif /* CONFIG_CMDLINE */

pr_debug("Command line is: %s\n", (char*)data);

/* break now */
return 1;
}


\end{lstlisting}




\begin{lstlisting}
/**
 * early_init_dt_scan_root - fetch the top level address and size cells
 */
int __init early_init_dt_scan_root(unsigned long node, const char *uname,
				   int depth, void *data)
{
	const __be32 *prop;

	if (depth != 0)
		return 0;

	dt_root_size_cells = OF_ROOT_NODE_SIZE_CELLS_DEFAULT;
	dt_root_addr_cells = OF_ROOT_NODE_ADDR_CELLS_DEFAULT;

	prop = of_get_flat_dt_prop(node, "#size-cells", NULL);
	if (prop)
		dt_root_size_cells = be32_to_cpup(prop);
	pr_debug("dt_root_size_cells = %x\n", dt_root_size_cells);

	prop = of_get_flat_dt_prop(node, "#address-cells", NULL);
	if (prop)
		dt_root_addr_cells = be32_to_cpup(prop);
	pr_debug("dt_root_addr_cells = %x\n", dt_root_addr_cells);

	/* break now */
	return 1;
}
\end{lstlisting}



\begin{lstlisting}
 
/**
* early_init_dt_scan_memory - Look for an parse memory nodes
*/
int __init early_init_dt_scan_memory(unsigned long node, const char *uname,
                    int depth, void *data)
{
   const char *type = of_get_flat_dt_prop(node, "device_type", NULL);
   const __be32 *reg, *endp;
   int l;

   /* We are scanning "memory" nodes only */
   if (type == NULL) {
       /*
        * The longtrail doesn't have a device_type on the
        * /memory node, so look for the node called /memory@0.
        */
       if (!IS_ENABLED(CONFIG_PPC32) || depth != 1 || strcmp(uname, "memory@0") != 0)
           return 0;
   } else if (strcmp(type, "memory") != 0)
       return 0;

   reg = of_get_flat_dt_prop(node, "linux,usable-memory", &l);
   if (reg == NULL)
       reg = of_get_flat_dt_prop(node, "reg", &l);
   if (reg == NULL)
       return 0;

   endp = reg + (l / sizeof(__be32));

   pr_debug("memory scan node %s, reg size %d,\n", uname, l);

   while ((endp - reg) >= (dt_root_addr_cells + dt_root_size_cells)) {
       u64 base, size;

       base = dt_mem_next_cell(dt_root_addr_cells, &reg);
       size = dt_mem_next_cell(dt_root_size_cells, &reg);

       if (size == 0)
           continue;
       pr_debug(" - %llx ,  %llx\n", (unsigned long long)base,
           (unsigned long long)size);

       early_init_dt_add_memory_arch(base, size);
   }

   return 0;
}   
\end{lstlisting}