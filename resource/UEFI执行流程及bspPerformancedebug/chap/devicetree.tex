\chapter{Device Tree}

\section{在kernel启动时}

在 kernel 的 main.c 中的 setup\_arch ,在设置cpu 后, 
进行设备树的 设置, 如果 返回 NULL 
则会去尝试解析  tags.
\begin{lstlisting}
mdesc = setup_machine_fdt(__atags_pointer);
if (!mdesc)
	mdesc = setup_machine_tags(__atags_pointer, __machine_arch_type);
    machine_desc = mdesc;
	machine_name = mdesc->name;
\end{lstlisting}



接下来的 一些 文字我希望分析 启动过程中 和 device tree 相关的事情。现在看来 是要 和 
耗上了。




从注释上看 这个函数是要在  dtb 传入 内核的时候,来根据dtb 做一些相应设置的。  


在一开始 ,  如果发现 dt\_phys 为空, 即没有 dtb 传递进来, 那么,直接返回NULL , 返回后会又 tags 的 方法去 用 tags 的 方法进行设置。 
\begin{lstlisting}
early_init_dt_verify(phys_to_virt(dt_phys)
\end{lstlisting}
上面的 code 先 验证 device tree 是不是有效。 然后用  dtb 的 虚拟地址 赋值为 \label{initial\_boot\_params} .





接下来:

\begin{lstlisting}

static const void * __init arch_get_next_mach(const char *const **match)
{
	static const struct machine_desc *mdesc = __arch_info_begin;
	const struct machine_desc *m = mdesc;

	if (m >= __arch_info_end)
		return NULL;

	mdesc++;
	*match = m->dt_compat;
	return m;
}


mdesc = of_flat_dt_match_machine(mdesc_best, arch_get_next_mach);
\end{lstlisting}

of\_flat\_dt\_match\_machine从 表中找到 匹配的 machine . 传入的参数是 :
\begin{itemize}
    \item default\_match:  这个位置传入的是 mdesc\_best  如果没有在匹配表中找到 匹配的,就会用这个
    返回。
    \item get\_next\_compat : 传入的是arch\_get\_next\_mach  回调函数, 这个回调函数
    用来找到下一个 合适的 匹配表。
    
\end{itemize}


arch\_get\_next\_mach 中 \_\_arch\_info\_begin 到 \_\_arch\_info\_end 依次 返回machine\_desc 的 dt\_compat
成员。那么 这个区段 究竟 存了些什么, 谁 存的 呢? 接下来 就要把 这件事情搞得很明白。


在链接脚本中,存在这样的区段:
\begin{lstlisting}
    .init.arch.info : {
        __arch_info_begin = .;
        *(.arch.info.init)
        __arch_info_end = .;
    }  
\end{lstlisting}

从这里可以得知, 为了进行接下来的研究, 需要找到  把 attribute 设置为 在 section ".arch.info.init" 的代码。

而 这段代码为 :
\begin{lstlisting}
    arch/arm/include/asm/mach/arch.h

    /*
 * Set of macros to define architecture features.  This is built into
 * a table by the linker.
 */
#define MACHINE_START(_type,_name)			\
static const struct machine_desc __mach_desc_##_type	\
 __used							\
 __attribute__((__section__(".arch.info.init"))) = {	\
	.nr		= MACH_TYPE_##_type,		\
	.name		= _name,

#define MACHINE_END				\
};

#define DT_MACHINE_START(_name, _namestr)		\
static const struct machine_desc __mach_desc_##_name	\
 __used							\
 __attribute__((__section__(".arch.info.init"))) = {	\
	.nr		= ~0,				\
	.name		= _namestr,

#endif

\end{lstlisting}



在 i.mx6ull 中 有:
\begin{lstlisting}
    arch/arm/mach-imx/mach-imx6ul.c :

    static const char *imx6ul_dt_compat[] __initconst = {
        "fsl,imx6ul",
        "fsl,imx6ull",
        NULL,
    };
    
    DT_MACHINE_START(IMX6UL, "Freescale i.MX6 Ultralite (Device Tree)")
        .map_io		= imx6ul_map_io,
        .init_irq	= imx6ul_init_irq,
        .init_machine	= imx6ul_init_machine,
        .init_late	= imx6ul_init_late,
        .dt_compat	= imx6ul_dt_compat,
    MACHINE_END
      
\end{lstlisting}


看到这些后, 我们返回来 分析 of\_flat\_dt\_match\_machine 中的 逻辑。dt\_root 代表了
设备树的根节点。   接下来 , 依次 从 .arch.info.init 段中,取出 machine\_desc 结构,然后 
提取出 其中的 dt\_compat 成员 和 传入的dtb 中的 比对。 

具体的匹配过程:

\begin{lstlisting}

    /**
    * of_fdt_match - Return true if node matches a list of compatible values
    */
   int of_fdt_match(const void *blob, unsigned long node,
                    const char *const *compat)
   {
       unsigned int tmp, score = 0;
   
       if (!compat)
           return 0;
   
       while (*compat) {
           tmp = of_fdt_is_compatible(blob, node, *compat);
           if (tmp && (score == 0 || (tmp < score)))
               score = tmp;
           compat++;
       }
   
       return score;
   }

/**
 * of_flat_dt_match - Return true if node matches a list of compatible values
 */
int __init of_flat_dt_match(unsigned long node, const char *const *compat)
{
	return of_fdt_match(initial_boot_params, node, compat);
}
\end{lstlisting}

还记得 initial\_boot\_params 吗? 不记得请 \ref{点击}  , of\_flat\_dt\_match 根据 compat 从跟节点 
判断 dtb 中的 compatible 字段 是否最合适。 这里判断是不是合适 使用了 score 的方法, 当完全匹配时,
score 为 1  并返回; 否则

