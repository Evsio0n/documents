\section{workflow}

%------------------------------------------------
\begin{frame}[fragile]{Demo工作流程:两个指令}

\begin{enumerate}
\item 主机,从机上电。 外设驱动初始化。从机进入PWM 捕获状态,等待事件的发生。
\item 主机在接收到PC的 "begin"指令的时候。
  \begin{itemize}
    \item a.先转发给从机
    \item b. 开始 timer3 本地计数(local counter).
    \item c. 开始使用timer2 产生PWM 波。
  \end{itemize}
\item 当从机接收到"begin" 指令,且从机的timer2 捕获到clk上的第一个上升沿。
  \begin{itemize}
    \item a.  timer3 本地计数(local counter).
    \item b. 进入PWM 捕获同步状态。
  \end{itemize}
\item "report" 指令:
\begin{itemize}
  \item a. 主机接收到"report"指令后,上报自己的时间戳给PC。
  \item b.从机接收到"report"指令后,上报自己的时间戳给主机,主机转发给PC。
\end{itemize}

\end{enumerate}


\end{frame}




%------------------------------------------------
\begin{frame}[fragile]{定时器的应用}

主从机中:
\begin{itemize}
  \item timer2 的时钟产生或捕获PWM ,用于同步。是BAN系统中的大粒度(8ms)时钟。
  \item timer3 的时钟仅用于小粒度(100us)的 时间戳 测量。
\end{itemize}


对于从机来说,每捕获到一个上升沿 ,为了时间同步,需要强制使从机的jiffies,jiffies\_pwm增加一个捕获
周期(8ms)。


\end{frame}




%------------------------------------------------
\begin{frame}[fragile]{Demo工作流程:同步细节}

local\_tm\_t 为主从机代码中 描述时间的类型:
\begin{lstlisting}
typedef struct local_time_struct{
 unsigned char sec:7;   //秒
 unsigned char min:7;   //分
 unsigned char hour:5;  //时
 unsigned int  day;     //天
 unsigned int  jiffies; //本地时间戳
 unsigned int  jiffies_pwm;   //pwm 时间戳
} local_tm_t;
\end{lstlisting}


\begin{itemize}
  \item 主机使用timer2 产生的PWM 波 来作为同步信号,从机捕获pwm 上升沿。这个方波上升沿间隔固定且精确为4ms 或者 8ms,可以客制化,在方波的上升沿,会去触发jiffies\_pwm 累加。
  \item 主从都使用timer3 每隔100us 产生一个中断,  在中断中 jiffies 累加,到10000(1s)超时进位。作为本地时钟。
\end{itemize}

\end{frame}



%------------------------------------------------
\begin{frame}[fragile]{从机上升沿捕获中断}

tick\_sync 每隔 8ms 会被调用一次 ,用于同步本地时间戳:
\begin{lstlisting}
/* every 8ms 80*100us  */
void tick_sync(void)
{
  if( dev.local_tm.jiffies_pwm + 80 < 10000  ){
  dev.local_tm.jiffies_pwm = dev.local_tm.jiffies_pwm  + 80;
  }
  if( dev.local_tm.jiffies_pwm + 80 >= 10000  ){
    dev.evt.sec_need_update=1;
    dev.local_tm.jiffies_pwm = 80 - (10000 - dev.local_tm.jiffies_pwm);
  }
  //同步
  dev.local_tm.jiffies = dev.local_tm.jiffies_pwm;		//强等 同步
}
\end{lstlisting}

\end{frame}



%------------------------------------------------
\begin{frame}[fragile]{本地计数超时处理}

tick\_sync 每隔 100us 会被调用一次 ,用于小粒度计数:
\begin{lstlisting}
/*
  每100us 进入这个中断一次。
*/
void Tick(void)
{
  dev.local_tm.jiffies ++;
  if( dev.local_tm.jiffies % 80 == 0){
   //GPIO A5 产生 方波 用于和 同步线的方波做对比
   HAL_GPIO_WritePin(GPIOA,GPIO_PIN_5,1-HAL_GPIO_ReadPin(GPIOA,GPIO_PIN_5));
 }
  /* one second passed */
  if(dev.local_tm.jiffies >= 10000 ){
   dev.local_tm.jiffies  = 0;
 }
}
\end{lstlisting}

这里 从机的GPIOA5 有相对于 时钟同步线的方波输出。 是一个测试点。
\end{frame}
