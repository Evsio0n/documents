\section{workflow}

%------------------------------------------------
\begin{frame}[fragile]{Demo工作流程:两个指令}

\begin{enumerate}
\item 主机,从机上电。 外设驱动初始化。从机启PWM 捕获状态,等待事件的发生。
\item 主机在接收到PC的 "begin"指令的时候。
  \begin{itemize}
    \item a.先转发给从机
    \item b. 开始 timer3 本地计数(local counter).
    \item c. 开始使用timer2 产生PWM 波。
  \end{itemize}
\item 当从机接收到"begin" 指令,且从机的timer2 捕获到clk上的第一个上升沿。
  \begin{itemize}
    \item a.  timer3 本地计数(local counter).
    \item b. 进入PWM 捕获同步状态。
  \end{itemize}
\item "report" 指令:
\begin{itemize}
  \item a. 主机接收到"report"指令后,上报自己的时间戳给PC。
  \item b.从机接收到"report"指令后,上报自己的时间戳给主机,主机转发给PC。
\end{itemize}

\end{enumerate}


\end{frame}




%------------------------------------------------
\begin{frame}[fragile]{Demo工作流程:同步细节}

local\_tm\_t 为主从机代码中 描述时间的类型:
\begin{lstlisting}
typedef struct local_time_struct{
 unsigned char sec:7;   //秒
 unsigned char min:7;   //分
 unsigned char hour:5;  //时
 unsigned int  day;     //天
 unsigned int  jiffies; //本地时间戳
 unsigned int  jiffies_pwm;   //pwm 时间戳
} local_tm_t;
\end{lstlisting}


\begin{itemize}
  \item 主机使用timer2 产生的PWM 波 来作为同步信号,从机捕获pwm 上升沿。这个方波上升沿间隔固定且精确为4ms 或者 8ms,可以客制化,在方波的上升沿,会去触发jiffies\_pwm 累加。
  \item 主从都使用timer3 每隔100us 产生一个中断,  在中断中 jiffies 累加,到10000(1s)超时进位。作为本地时钟。
\end{itemize}

\end{frame}




%------------------------------------------------
\begin{frame}[fragile]{Demo工作流程:同步细节}

timer2 的时钟用于产生或捕获PWM ,这个时钟用于同步。

timer3 的时钟仅用于小粒度的 时间戳 测量。

对于从机来说,每捕获到一个上升沿 ,为了时间同步,需要强制使从机的jiffies归零,jiffies\_pwm增加一个捕获
周期(4ms或者8ms)。


\end{frame}
